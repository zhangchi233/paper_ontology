\section*{Discussion}
\label{sec:discussion}
\subsection*{Pros and Cons of the LLM-driven Ontology-based Safety Management System}
\subsubsection{Pros}
The LLM-driven ontology-based safety management system has several advantages.
Firstly, compared with traditional ontology-based safety management systems, 
the LLM-driven system can automatically generate ontology from unstructured text data and images, which saves time and labor costs.
Moreover, traditional ontology-based safety management systems lacks the ability to quantitatively evaluate the relation between 
different concepts in the ontology, for example, the relation between the renovation activity and the potential hazards not quantitatively evaluated.
The LLM-driven system provides a methods to integrate data-driven methods to assign weights to the relation between risks and regulations,
which can help to evaluate the potential hazards more accurately.

Secondly, the LLM-driven system can automatically and dynamically update the ontology based on the new data, which ensures the system's scalability and adaptability.
Some details of the renovation activities may be changed during the renovation process, such as the regulation of construction equipment, the safety measures, etc.
Traditional ontology-based safety management system may need to manually update the ontology, which is time-consuming and error-prone.
The LLM-driven system can automatically update the ontology based on the new data, which ensures the system's scalability and adaptability.

Moreover, the LLM-driven system can provide a more user-friendly interface for safety management in renovation activities.
The traditional ontology-based safety management system may require the user to have a professional background in ontology and safety management.
The user needs to conduct a series of operation on ontology platform such as Protégé to query the ontology and get the results.
The LLM-driven system can provide a more user-friendly interface, such as a QA system, which can interact with the user in natural language.
The LLM is also more objective and less biased than human experts, which can provide more accurate and reliable results.

Compared with other multi-modal methods, the LLM-driven ontology-based safety management system show a better performance
\subsubsection{Cons}
There are also some limitations of the LLM-driven ontology-based safety management system.

Firstly, our system needs enough data support to build the ontology.